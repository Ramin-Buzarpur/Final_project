% Ramin Buzarpur
% student id : 402541206
% final project 

\documentclass[12pt]{article}
\usepackage[utf8]{inputenc}
\usepackage{listings}
\usepackage{xcolor}
\usepackage{hyperref}
\usepackage{graphicx}
\lstdefinelanguage{yaml}{
    keywords={true,false,null,y,n},
    keywordstyle=\color{blue},
    basicstyle=\ttfamily\small,
    sensitive=false,
    comment=[l]{\#},
    commentstyle=\color{gray},
    stringstyle=\color{red},
    morestring=[b]',
    morestring=[b]"
}

\title{Git, GitHub, Vim, and Bash}
\author{}
\date{}

\begin{document}

\maketitle

\section*{Table of Contents}
\begin{enumerate}
    \item \textbf{1. Git and GitHub}
    \begin{enumerate}
        \item 1.1 Repository Initialization and Commits
        \item 1.2 GitHub Actions for LaTeX Compilation
    \end{enumerate}
    \item \textbf{2. Exploration Tasks}
    \begin{enumerate}
        \item 2.1 Vim Advanced Features
        \item 2.2 Memory Profiling
        \begin{enumerate}
            \item 2.2.1 Memory Leak
            \item 2.2.2 Memory Profilers (Valgrind)
        \end{enumerate}
        \item 2.3 GNU/Linux Bash Scripting
        \begin{enumerate}
            \item 2.3.1 fzf
            \item 2.3.2 Using fzf to Find Your Favorite PDF
            \item 2.3.3 Opening the File Using Zathura
        \end{enumerate}
    \end{enumerate}
    \item \textbf{3. Git and FOSS}
    \begin{enumerate}
        \item 3.1 README.md
        \item 3.2 Issues
        \item 3.3 FOSS Contribution
    \end{enumerate}
    \item \textbf{Conclusion}
\end{enumerate}

\newpage

\section*{1.1 Setting up a Repository and Committing Changes}
Setting up a Git repository involves several steps:
\begin{enumerate}
    \item \textbf{Installing Git:}
    Install Git on your system if it’s not already installed:
    \begin{itemize}
        \item Linux:
        \begin{lstlisting}[language=bash]
sudo apt update
sudo apt install git
        \end{lstlisting}
        \item Windows/macOS: Download Git from \url{https://git-scm.com}.
    \end{itemize}
    \item \textbf{Initializing a Repository:}
    Navigate to your project directory and initialize a repository:
    \begin{lstlisting}[language=bash]
cd /path/to/project
git init
    \end{lstlisting}
    \item \textbf{Configuring Git:}
    Set your username and email:
    \begin{lstlisting}[language=bash]
git config --global user.name "Your Name"
git config --global user.email "your-email@example.com"
    \end{lstlisting}
    \item \textbf{Staging and Committing Changes:}
    Add and commit files:
    \begin{lstlisting}[language=bash]
git add .
git commit -m "Initial commit"
    \end{lstlisting}
    \item \textbf{Connecting to GitHub:}
    Add a remote and push changes:
    \begin{lstlisting}[language=bash]
git remote add origin https://github.com/username/repository.git
git push -u origin master
    \end{lstlisting}
\end{enumerate}
\end{document}